\documentclass[18pt,a4paper]{article} 
\usepackage{amsmath}
\usepackage{graphicx}

\title{$J/\psi$ analysis note} 
\author{Qian Yang} 
\date{29 Jun 2014}

\begin{document}
\maketitle

\section{Introduction} 
Quantum Chromodynamics(QCD) predicts a phase transition from hadronic matter to a partonic phase of matter,  known as Quark-Gluon Plasma(QGP), at high energy density an temperature. Ultra-relativistic heavy ion collisions provide a unique tool to create and study this strongly interacting matter.  Many QGP signatures have been proposed which include rare probes(e.g. direct photon and dilepton production and jet modification) as well as bulk probes(e.g. enhanced strangeness and anti baryon production and strong collective). Compared to bulk probes, rare probes are more robust, since bulk probes are most probably disguised or diluted by other processes such as the final state interaction. So a significant part of the extensive experimental heavy-in programs dedicated to measuring quarkonium yields. Quarkonium suppression in heavy-ion collisions thought to be a demonstration of the medium. However, not all of the observed quarkonium suppression in nucleus-nucleus collisions relative to scaled proton-proton collisions is due to quark-gluon plasma formation. Cold-nuclear-matter effects (initial-state effects , absorption) also introduce quarkonuim suppression. Despite of the indirectly measurement of the gluon density, which makes a large range of possible shadowing effects, the seems rather straightforward absorption effect also need additional information about the quarkonium, since it is unknown whether the object traversing the nucleus is a precursor color-octet state or a fully formed  color-signlet quarkonium state. If it is an octet stat, it is assumed to immediately interact with a large,  finite cross section since it is a colored object.  If it is produced as a small color singlet, the absorption cross section immediately after the production of the $Q\bar{Q}$ pair should be small and increasing with proper time until, at the formation time, it reaches its final-state size. so it is better to have a good understanding of the production mechanism, especially in p+p collisions, which include direct production via gluon fusion, parton fragmentation, and feed down from excited quarkonium states. 
However, the quarkonium production mechanism still remains an open question to date after decades of efforts. New quarkonium measurements, especially production at high tranverse momentum . In this analysis  we report on the new measurements of $J/\psi$ and $\psi(2S)$ invariant yields in a broad range of transverse momentum ($4<p_T<20 GeV/c$ ) at mid-rapidity ($\|y\| < 1.0$) in p+p collisions at $ \sqrt{s}= 500 GeV$ from STAR based on Run11 datasets with a HT trigger. Comparisons among model calculations will be discussed.

\section{Data sets and Triggers} 
In this analysis we use the High Tower(HT) triggered events to reconstruct high $p_T$ $J/\psi$. During 2011 500GeV p+p runs ,there were 3 HT triggers, Their trigger name,trigger Id,sampled luminosity, number of events and the ET threshold are listed in Table1. The recorded number of events for the BHT trigger in Au + Au collisions are shown in Figure.  The data was produced in the production series P11id using SL11d library. Due to limited bandwidth, only a fraction of the events satisfying HT1 triggers can be recorded at the end of runs. but for HT0 data a VPD coincidence from East and West is applied, which scaled down the trigger rate and also increased the vertex resolution. HT2 has a higher threshold than HT1 and HT0, so to recorded as much integrated luminosity as possible the VPD coincidence requirement  is removed. 
\begin{figure}
\centering
\includegraphics[scale=0.3]{lum_perday_BHT1.png}
\caption{vetrex Z}
\label{fig1}
\end{figure}

\section{Event Selection}
The importance of  the vertex-finding algorithm to find a real trigger event is self-evident. In p+p collisions, the low charged-particle multiplicity and high luminosity, which result a pileup rich collisions, are all set obstacle to find the truth vertex. the charged-particle multiplicity can be seen in Fig.2. A PPV vertex finer(Pile-up Protected Vertex finder) was used in the data. PPV,  which finds the Z position of a vertex, requires a beam-line constraint to determine the x and y values of the vertex. The beam line constraint is calculated by fitting high multiplicity events with Minuit VF without any constraints on the  vertex position. A straight line is fit to the vertex distributions to obtain a relationship between x, y and z. This method will give a correct vertices,  we take the highest ranking vertex in every event as our event primary vertex. Of cause a valid vertex requirement,3 components of the  reconstructed vertex positions not simultaneously below $10^-5$cm. is applied too.Fig.3. is  the $V_x$ vs. $V_y$ distribution, the beam line is clearly shown. Fig.4. , Fig5. and Fig.6 show the $V_z$ distribution in BHT0*MB, BHT1 and BHT2, separately.  as we mention above the BHT0 with a VPD  East  and West coincidence will have a better $V_z$ resolution.  All of the cuts on event level for the triggers are summarized it Table 2.
\subsection{Track Selection}
 
\subsubsection{Subsubsection} 
 This is sample subsubsection.
 
 \section{$dE/dx$}
 EID
\subsection{Subsubsection} 
 This is sample subsubsection.

\paragraph{Paragraph} 
 This is sample paragraph.
\begin{thebibliography}{00}
\bibitem{1} 
 This is sample bibitem one. 
 \bibitem{2} 
 This is sample bibitem two. 
 \bibitem{3} 
 This is sample bibitem three.
\end{thebibliography} 
 \end{document}
